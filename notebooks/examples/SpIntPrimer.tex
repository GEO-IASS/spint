
% Default to the notebook output style

    


% Inherit from the specified cell style.




    
\documentclass[11pt]{article}

    
    
    \usepackage[T1]{fontenc}
    % Nicer default font than Computer Modern for most use cases
    \usepackage{palatino}

    % Basic figure setup, for now with no caption control since it's done
    % automatically by Pandoc (which extracts ![](path) syntax from Markdown).
    \usepackage{graphicx}
    % We will generate all images so they have a width \maxwidth. This means
    % that they will get their normal width if they fit onto the page, but
    % are scaled down if they would overflow the margins.
    \makeatletter
    \def\maxwidth{\ifdim\Gin@nat@width>\linewidth\linewidth
    \else\Gin@nat@width\fi}
    \makeatother
    \let\Oldincludegraphics\includegraphics
    % Set max figure width to be 80% of text width, for now hardcoded.
    \renewcommand{\includegraphics}[1]{\Oldincludegraphics[width=.8\maxwidth]{#1}}
    % Ensure that by default, figures have no caption (until we provide a
    % proper Figure object with a Caption API and a way to capture that
    % in the conversion process - todo).
    \usepackage{caption}
    \DeclareCaptionLabelFormat{nolabel}{}
    \captionsetup{labelformat=nolabel}

    \usepackage{adjustbox} % Used to constrain images to a maximum size 
    \usepackage{xcolor} % Allow colors to be defined
    \usepackage{enumerate} % Needed for markdown enumerations to work
    \usepackage{geometry} % Used to adjust the document margins
    \usepackage{amsmath} % Equations
    \usepackage{amssymb} % Equations
    \usepackage{textcomp} % defines textquotesingle
    % Hack from http://tex.stackexchange.com/a/47451/13684:
    \AtBeginDocument{%
        \def\PYZsq{\textquotesingle}% Upright quotes in Pygmentized code
    }
    \usepackage{upquote} % Upright quotes for verbatim code
    \usepackage{eurosym} % defines \euro
    \usepackage[mathletters]{ucs} % Extended unicode (utf-8) support
    \usepackage[utf8x]{inputenc} % Allow utf-8 characters in the tex document
    \usepackage{fancyvrb} % verbatim replacement that allows latex
    \usepackage{grffile} % extends the file name processing of package graphics 
                         % to support a larger range 
    % The hyperref package gives us a pdf with properly built
    % internal navigation ('pdf bookmarks' for the table of contents,
    % internal cross-reference links, web links for URLs, etc.)
    \usepackage{hyperref}
    \usepackage{longtable} % longtable support required by pandoc >1.10
    \usepackage{booktabs}  % table support for pandoc > 1.12.2
    \usepackage[normalem]{ulem} % ulem is needed to support strikethroughs (\sout)
                                % normalem makes italics be italics, not underlines
    

    
    
    % Colors for the hyperref package
    \definecolor{urlcolor}{rgb}{0,.145,.698}
    \definecolor{linkcolor}{rgb}{.71,0.21,0.01}
    \definecolor{citecolor}{rgb}{.12,.54,.11}

    % ANSI colors
    \definecolor{ansi-black}{HTML}{3E424D}
    \definecolor{ansi-black-intense}{HTML}{282C36}
    \definecolor{ansi-red}{HTML}{E75C58}
    \definecolor{ansi-red-intense}{HTML}{B22B31}
    \definecolor{ansi-green}{HTML}{00A250}
    \definecolor{ansi-green-intense}{HTML}{007427}
    \definecolor{ansi-yellow}{HTML}{DDB62B}
    \definecolor{ansi-yellow-intense}{HTML}{B27D12}
    \definecolor{ansi-blue}{HTML}{208FFB}
    \definecolor{ansi-blue-intense}{HTML}{0065CA}
    \definecolor{ansi-magenta}{HTML}{D160C4}
    \definecolor{ansi-magenta-intense}{HTML}{A03196}
    \definecolor{ansi-cyan}{HTML}{60C6C8}
    \definecolor{ansi-cyan-intense}{HTML}{258F8F}
    \definecolor{ansi-white}{HTML}{C5C1B4}
    \definecolor{ansi-white-intense}{HTML}{A1A6B2}

    % commands and environments needed by pandoc snippets
    % extracted from the output of `pandoc -s`
    \providecommand{\tightlist}{%
      \setlength{\itemsep}{0pt}\setlength{\parskip}{0pt}}
    \DefineVerbatimEnvironment{Highlighting}{Verbatim}{commandchars=\\\{\}}
    % Add ',fontsize=\small' for more characters per line
    \newenvironment{Shaded}{}{}
    \newcommand{\KeywordTok}[1]{\textcolor[rgb]{0.00,0.44,0.13}{\textbf{{#1}}}}
    \newcommand{\DataTypeTok}[1]{\textcolor[rgb]{0.56,0.13,0.00}{{#1}}}
    \newcommand{\DecValTok}[1]{\textcolor[rgb]{0.25,0.63,0.44}{{#1}}}
    \newcommand{\BaseNTok}[1]{\textcolor[rgb]{0.25,0.63,0.44}{{#1}}}
    \newcommand{\FloatTok}[1]{\textcolor[rgb]{0.25,0.63,0.44}{{#1}}}
    \newcommand{\CharTok}[1]{\textcolor[rgb]{0.25,0.44,0.63}{{#1}}}
    \newcommand{\StringTok}[1]{\textcolor[rgb]{0.25,0.44,0.63}{{#1}}}
    \newcommand{\CommentTok}[1]{\textcolor[rgb]{0.38,0.63,0.69}{\textit{{#1}}}}
    \newcommand{\OtherTok}[1]{\textcolor[rgb]{0.00,0.44,0.13}{{#1}}}
    \newcommand{\AlertTok}[1]{\textcolor[rgb]{1.00,0.00,0.00}{\textbf{{#1}}}}
    \newcommand{\FunctionTok}[1]{\textcolor[rgb]{0.02,0.16,0.49}{{#1}}}
    \newcommand{\RegionMarkerTok}[1]{{#1}}
    \newcommand{\ErrorTok}[1]{\textcolor[rgb]{1.00,0.00,0.00}{\textbf{{#1}}}}
    \newcommand{\NormalTok}[1]{{#1}}
    
    % Additional commands for more recent versions of Pandoc
    \newcommand{\ConstantTok}[1]{\textcolor[rgb]{0.53,0.00,0.00}{{#1}}}
    \newcommand{\SpecialCharTok}[1]{\textcolor[rgb]{0.25,0.44,0.63}{{#1}}}
    \newcommand{\VerbatimStringTok}[1]{\textcolor[rgb]{0.25,0.44,0.63}{{#1}}}
    \newcommand{\SpecialStringTok}[1]{\textcolor[rgb]{0.73,0.40,0.53}{{#1}}}
    \newcommand{\ImportTok}[1]{{#1}}
    \newcommand{\DocumentationTok}[1]{\textcolor[rgb]{0.73,0.13,0.13}{\textit{{#1}}}}
    \newcommand{\AnnotationTok}[1]{\textcolor[rgb]{0.38,0.63,0.69}{\textbf{\textit{{#1}}}}}
    \newcommand{\CommentVarTok}[1]{\textcolor[rgb]{0.38,0.63,0.69}{\textbf{\textit{{#1}}}}}
    \newcommand{\VariableTok}[1]{\textcolor[rgb]{0.10,0.09,0.49}{{#1}}}
    \newcommand{\ControlFlowTok}[1]{\textcolor[rgb]{0.00,0.44,0.13}{\textbf{{#1}}}}
    \newcommand{\OperatorTok}[1]{\textcolor[rgb]{0.40,0.40,0.40}{{#1}}}
    \newcommand{\BuiltInTok}[1]{{#1}}
    \newcommand{\ExtensionTok}[1]{{#1}}
    \newcommand{\PreprocessorTok}[1]{\textcolor[rgb]{0.74,0.48,0.00}{{#1}}}
    \newcommand{\AttributeTok}[1]{\textcolor[rgb]{0.49,0.56,0.16}{{#1}}}
    \newcommand{\InformationTok}[1]{\textcolor[rgb]{0.38,0.63,0.69}{\textbf{\textit{{#1}}}}}
    \newcommand{\WarningTok}[1]{\textcolor[rgb]{0.38,0.63,0.69}{\textbf{\textit{{#1}}}}}
    
    
    % Define a nice break command that doesn't care if a line doesn't already
    % exist.
    \def\br{\hspace*{\fill} \\* }
    % Math Jax compatability definitions
    \def\gt{>}
    \def\lt{<}
    % Document parameters
    \title{SpIntPrimer}
    
    
    

    % Pygments definitions
    
\makeatletter
\def\PY@reset{\let\PY@it=\relax \let\PY@bf=\relax%
    \let\PY@ul=\relax \let\PY@tc=\relax%
    \let\PY@bc=\relax \let\PY@ff=\relax}
\def\PY@tok#1{\csname PY@tok@#1\endcsname}
\def\PY@toks#1+{\ifx\relax#1\empty\else%
    \PY@tok{#1}\expandafter\PY@toks\fi}
\def\PY@do#1{\PY@bc{\PY@tc{\PY@ul{%
    \PY@it{\PY@bf{\PY@ff{#1}}}}}}}
\def\PY#1#2{\PY@reset\PY@toks#1+\relax+\PY@do{#2}}

\expandafter\def\csname PY@tok@gd\endcsname{\def\PY@tc##1{\textcolor[rgb]{0.63,0.00,0.00}{##1}}}
\expandafter\def\csname PY@tok@gu\endcsname{\let\PY@bf=\textbf\def\PY@tc##1{\textcolor[rgb]{0.50,0.00,0.50}{##1}}}
\expandafter\def\csname PY@tok@gt\endcsname{\def\PY@tc##1{\textcolor[rgb]{0.00,0.27,0.87}{##1}}}
\expandafter\def\csname PY@tok@gs\endcsname{\let\PY@bf=\textbf}
\expandafter\def\csname PY@tok@gr\endcsname{\def\PY@tc##1{\textcolor[rgb]{1.00,0.00,0.00}{##1}}}
\expandafter\def\csname PY@tok@cm\endcsname{\let\PY@it=\textit\def\PY@tc##1{\textcolor[rgb]{0.25,0.50,0.50}{##1}}}
\expandafter\def\csname PY@tok@vg\endcsname{\def\PY@tc##1{\textcolor[rgb]{0.10,0.09,0.49}{##1}}}
\expandafter\def\csname PY@tok@vi\endcsname{\def\PY@tc##1{\textcolor[rgb]{0.10,0.09,0.49}{##1}}}
\expandafter\def\csname PY@tok@mh\endcsname{\def\PY@tc##1{\textcolor[rgb]{0.40,0.40,0.40}{##1}}}
\expandafter\def\csname PY@tok@cs\endcsname{\let\PY@it=\textit\def\PY@tc##1{\textcolor[rgb]{0.25,0.50,0.50}{##1}}}
\expandafter\def\csname PY@tok@ge\endcsname{\let\PY@it=\textit}
\expandafter\def\csname PY@tok@vc\endcsname{\def\PY@tc##1{\textcolor[rgb]{0.10,0.09,0.49}{##1}}}
\expandafter\def\csname PY@tok@il\endcsname{\def\PY@tc##1{\textcolor[rgb]{0.40,0.40,0.40}{##1}}}
\expandafter\def\csname PY@tok@go\endcsname{\def\PY@tc##1{\textcolor[rgb]{0.53,0.53,0.53}{##1}}}
\expandafter\def\csname PY@tok@cp\endcsname{\def\PY@tc##1{\textcolor[rgb]{0.74,0.48,0.00}{##1}}}
\expandafter\def\csname PY@tok@gi\endcsname{\def\PY@tc##1{\textcolor[rgb]{0.00,0.63,0.00}{##1}}}
\expandafter\def\csname PY@tok@gh\endcsname{\let\PY@bf=\textbf\def\PY@tc##1{\textcolor[rgb]{0.00,0.00,0.50}{##1}}}
\expandafter\def\csname PY@tok@ni\endcsname{\let\PY@bf=\textbf\def\PY@tc##1{\textcolor[rgb]{0.60,0.60,0.60}{##1}}}
\expandafter\def\csname PY@tok@nl\endcsname{\def\PY@tc##1{\textcolor[rgb]{0.63,0.63,0.00}{##1}}}
\expandafter\def\csname PY@tok@nn\endcsname{\let\PY@bf=\textbf\def\PY@tc##1{\textcolor[rgb]{0.00,0.00,1.00}{##1}}}
\expandafter\def\csname PY@tok@no\endcsname{\def\PY@tc##1{\textcolor[rgb]{0.53,0.00,0.00}{##1}}}
\expandafter\def\csname PY@tok@na\endcsname{\def\PY@tc##1{\textcolor[rgb]{0.49,0.56,0.16}{##1}}}
\expandafter\def\csname PY@tok@nb\endcsname{\def\PY@tc##1{\textcolor[rgb]{0.00,0.50,0.00}{##1}}}
\expandafter\def\csname PY@tok@nc\endcsname{\let\PY@bf=\textbf\def\PY@tc##1{\textcolor[rgb]{0.00,0.00,1.00}{##1}}}
\expandafter\def\csname PY@tok@nd\endcsname{\def\PY@tc##1{\textcolor[rgb]{0.67,0.13,1.00}{##1}}}
\expandafter\def\csname PY@tok@ne\endcsname{\let\PY@bf=\textbf\def\PY@tc##1{\textcolor[rgb]{0.82,0.25,0.23}{##1}}}
\expandafter\def\csname PY@tok@nf\endcsname{\def\PY@tc##1{\textcolor[rgb]{0.00,0.00,1.00}{##1}}}
\expandafter\def\csname PY@tok@si\endcsname{\let\PY@bf=\textbf\def\PY@tc##1{\textcolor[rgb]{0.73,0.40,0.53}{##1}}}
\expandafter\def\csname PY@tok@s2\endcsname{\def\PY@tc##1{\textcolor[rgb]{0.73,0.13,0.13}{##1}}}
\expandafter\def\csname PY@tok@nt\endcsname{\let\PY@bf=\textbf\def\PY@tc##1{\textcolor[rgb]{0.00,0.50,0.00}{##1}}}
\expandafter\def\csname PY@tok@nv\endcsname{\def\PY@tc##1{\textcolor[rgb]{0.10,0.09,0.49}{##1}}}
\expandafter\def\csname PY@tok@s1\endcsname{\def\PY@tc##1{\textcolor[rgb]{0.73,0.13,0.13}{##1}}}
\expandafter\def\csname PY@tok@ch\endcsname{\let\PY@it=\textit\def\PY@tc##1{\textcolor[rgb]{0.25,0.50,0.50}{##1}}}
\expandafter\def\csname PY@tok@m\endcsname{\def\PY@tc##1{\textcolor[rgb]{0.40,0.40,0.40}{##1}}}
\expandafter\def\csname PY@tok@gp\endcsname{\let\PY@bf=\textbf\def\PY@tc##1{\textcolor[rgb]{0.00,0.00,0.50}{##1}}}
\expandafter\def\csname PY@tok@sh\endcsname{\def\PY@tc##1{\textcolor[rgb]{0.73,0.13,0.13}{##1}}}
\expandafter\def\csname PY@tok@ow\endcsname{\let\PY@bf=\textbf\def\PY@tc##1{\textcolor[rgb]{0.67,0.13,1.00}{##1}}}
\expandafter\def\csname PY@tok@sx\endcsname{\def\PY@tc##1{\textcolor[rgb]{0.00,0.50,0.00}{##1}}}
\expandafter\def\csname PY@tok@bp\endcsname{\def\PY@tc##1{\textcolor[rgb]{0.00,0.50,0.00}{##1}}}
\expandafter\def\csname PY@tok@c1\endcsname{\let\PY@it=\textit\def\PY@tc##1{\textcolor[rgb]{0.25,0.50,0.50}{##1}}}
\expandafter\def\csname PY@tok@o\endcsname{\def\PY@tc##1{\textcolor[rgb]{0.40,0.40,0.40}{##1}}}
\expandafter\def\csname PY@tok@kc\endcsname{\let\PY@bf=\textbf\def\PY@tc##1{\textcolor[rgb]{0.00,0.50,0.00}{##1}}}
\expandafter\def\csname PY@tok@c\endcsname{\let\PY@it=\textit\def\PY@tc##1{\textcolor[rgb]{0.25,0.50,0.50}{##1}}}
\expandafter\def\csname PY@tok@mf\endcsname{\def\PY@tc##1{\textcolor[rgb]{0.40,0.40,0.40}{##1}}}
\expandafter\def\csname PY@tok@err\endcsname{\def\PY@bc##1{\setlength{\fboxsep}{0pt}\fcolorbox[rgb]{1.00,0.00,0.00}{1,1,1}{\strut ##1}}}
\expandafter\def\csname PY@tok@mb\endcsname{\def\PY@tc##1{\textcolor[rgb]{0.40,0.40,0.40}{##1}}}
\expandafter\def\csname PY@tok@ss\endcsname{\def\PY@tc##1{\textcolor[rgb]{0.10,0.09,0.49}{##1}}}
\expandafter\def\csname PY@tok@sr\endcsname{\def\PY@tc##1{\textcolor[rgb]{0.73,0.40,0.53}{##1}}}
\expandafter\def\csname PY@tok@mo\endcsname{\def\PY@tc##1{\textcolor[rgb]{0.40,0.40,0.40}{##1}}}
\expandafter\def\csname PY@tok@kd\endcsname{\let\PY@bf=\textbf\def\PY@tc##1{\textcolor[rgb]{0.00,0.50,0.00}{##1}}}
\expandafter\def\csname PY@tok@mi\endcsname{\def\PY@tc##1{\textcolor[rgb]{0.40,0.40,0.40}{##1}}}
\expandafter\def\csname PY@tok@kn\endcsname{\let\PY@bf=\textbf\def\PY@tc##1{\textcolor[rgb]{0.00,0.50,0.00}{##1}}}
\expandafter\def\csname PY@tok@cpf\endcsname{\let\PY@it=\textit\def\PY@tc##1{\textcolor[rgb]{0.25,0.50,0.50}{##1}}}
\expandafter\def\csname PY@tok@kr\endcsname{\let\PY@bf=\textbf\def\PY@tc##1{\textcolor[rgb]{0.00,0.50,0.00}{##1}}}
\expandafter\def\csname PY@tok@s\endcsname{\def\PY@tc##1{\textcolor[rgb]{0.73,0.13,0.13}{##1}}}
\expandafter\def\csname PY@tok@kp\endcsname{\def\PY@tc##1{\textcolor[rgb]{0.00,0.50,0.00}{##1}}}
\expandafter\def\csname PY@tok@w\endcsname{\def\PY@tc##1{\textcolor[rgb]{0.73,0.73,0.73}{##1}}}
\expandafter\def\csname PY@tok@kt\endcsname{\def\PY@tc##1{\textcolor[rgb]{0.69,0.00,0.25}{##1}}}
\expandafter\def\csname PY@tok@sc\endcsname{\def\PY@tc##1{\textcolor[rgb]{0.73,0.13,0.13}{##1}}}
\expandafter\def\csname PY@tok@sb\endcsname{\def\PY@tc##1{\textcolor[rgb]{0.73,0.13,0.13}{##1}}}
\expandafter\def\csname PY@tok@k\endcsname{\let\PY@bf=\textbf\def\PY@tc##1{\textcolor[rgb]{0.00,0.50,0.00}{##1}}}
\expandafter\def\csname PY@tok@se\endcsname{\let\PY@bf=\textbf\def\PY@tc##1{\textcolor[rgb]{0.73,0.40,0.13}{##1}}}
\expandafter\def\csname PY@tok@sd\endcsname{\let\PY@it=\textit\def\PY@tc##1{\textcolor[rgb]{0.73,0.13,0.13}{##1}}}

\def\PYZbs{\char`\\}
\def\PYZus{\char`\_}
\def\PYZob{\char`\{}
\def\PYZcb{\char`\}}
\def\PYZca{\char`\^}
\def\PYZam{\char`\&}
\def\PYZlt{\char`\<}
\def\PYZgt{\char`\>}
\def\PYZsh{\char`\#}
\def\PYZpc{\char`\%}
\def\PYZdl{\char`\$}
\def\PYZhy{\char`\-}
\def\PYZsq{\char`\'}
\def\PYZdq{\char`\"}
\def\PYZti{\char`\~}
% for compatibility with earlier versions
\def\PYZat{@}
\def\PYZlb{[}
\def\PYZrb{]}
\makeatother


    % Exact colors from NB
    \definecolor{incolor}{rgb}{0.0, 0.0, 0.5}
    \definecolor{outcolor}{rgb}{0.545, 0.0, 0.0}



    
    % Prevent overflowing lines due to hard-to-break entities
    \sloppy 
    % Setup hyperref package
    \hypersetup{
      breaklinks=true,  % so long urls are correctly broken across lines
      colorlinks=true,
      urlcolor=urlcolor,
      linkcolor=linkcolor,
      citecolor=citecolor,
      }
    % Slightly bigger margins than the latex defaults
    
    \geometry{verbose,tmargin=1in,bmargin=1in,lmargin=1in,rmargin=1in}
    
    

    \begin{document}
    
    
    \maketitle
    
    

    
    \section{\texorpdfstring{A primer for working with the
\textbf{\emph{Sp}}atial \textbf{\emph{Int}}eraction modeling (SpInt)
module in the python spatial analysis library
(PySAL)}{A primer for working with the Spatial Interaction modeling (SpInt) module in the python spatial analysis library (PySAL)}}\label{a-primer-for-working-with-the-spatial-interaction-modeling-spint-module-in-the-python-spatial-analysis-library-pysal}

    \subsection{Introduction}\label{introduction}

    Spatial interaction modeling involves the analysis of flows from an
origin to a destination either over physical space (i.e., migration) or
through abstract space (i.e., telecommunication). Despite being a
fundamental technique to many geogaphic disciplines, there is relatively
little software available to carry out spatial interaction modeling and
the analysis of flow data. Therefore, the purpose of this primer is to
provide an overview of the recently develped spatial interaction
modeling (SpInt) module of the python spatial analysis library (PySAL).
First, the current framework of the module will be highlighted. Next,
the main functionality of the module will be illustrated with a common
regional science example, migration flows, with a dataset previously
used for spatial interaction modeling tutorials in the R programming
environment. Finally, some auxilliary functionality and future additions
are discussed.

    \subsection{The SpInt Framework}\label{the-spint-framework}

    \subsubsection{Modeling framework}\label{modeling-framework}

    The core purpose of the SpInt module is to provide the functionality to
calibrate spatial interaction models. Since the ``family'' of spatial
interaction models put forth by Wilson (Wilson, 1971) are perhaps the
most popular, they were chosen as the starting point of the module.
Consider the basic gravity model (Fotheringham \& O'Kelly, 1989),

\[T_{ij} = k\frac{V_{i}^\mu W_{j}^\alpha}{d_{ij}^\beta} \quad(1)\]

where

\begin{itemize}
\item
  \(T_{ij}\) = an \(n \times m\) matrix of flows between \(n\) origins
  (subscripted by \(i\)) to \(m\) destinations (subscripted by \(j\))
\item
  \(V\) = an \(n \times p\) and vector of \(p\) origin attributes
  describing the emissiveness of \(i\)
\item
  \(W\) = an \(m \times p\) vector of \(p\) destination attributes
  describing the attractiveness of \(j\)
\item
  \(d\) = an \(n \times m\) matrix of the costs to overcome the physical
  separation between \(i\) and \(j\) (usually distance or time)
\item
  \(k\) = a scaling factor to be estimated to ensure the total flows is
  consistent
\item
  \(\mu\) = a \(p \times 1\) vector of parameters representing the
  effect of \(p\) origin attributes on flows
\item
  \(\alpha\) = a \(p \times 1\) vector of parameters representing the
  effect of \(p\) destination attributes on flows
\item
  \(\beta\) = an exponential parameter representing the effect of
  movement costs on flows.
\end{itemize}

When data for \(T\), \(V\), \(W\), and \(d\) are available we can
estimate the model parameters (also called calibration), which summarize
the effect that each model component contributes towards explaining the
system of known flows (\(T\)). In contrast, known parameters can be used
to predict unknown flows when there are deviations in model components
(\(V\), \(W\), and \(d\)) or the set of locations in the system are
altered.

Using an entropy-maximizing framework, Wilson derives a more informative
and flexible ``family'' of four spatial interaction models (Wilson,
1971). This framework seeks to assign flows between a set of origins and
destinations by finding the most probable configuration of flows out of
all possible configurations, without making any additional assumptions.
By using a common optimization problem and including information about
the total inflows and outflows at each location (also called
constraints), the following ``family'' of models can be obtained:

\[
\begin{align}
&Unconstrained \ (Gravity) \\
&Tij = V_{i}^\mu W_{j}^\alpha  f(d_{ij}) \quad & (2) \\
\\
&Production-constrained \\
&T_{ij} = A_{i}O_{i}W_{j}^\alpha f(d_{ij}) \quad & (3) \\
&A_{i} = \sum_{j} W_{j}^\alpha f(d_{ij}) \quad & (3a) \\
\\
&Attraction-constrained \\
&T_{ij} = B_{j}D_{j}V_{i}^\mu f(d_{ij}) \quad & (4) \\
&B_{j} = \sum_{i} V_{i}^\mu f(d_{ij}) \quad & (4a) \\
\\
&Doubly-constrained \\
&T_{ij} = A_{i}B_{j}O_{i}D_{j}f(d_{ij}) \quad & (5) \\
&A_{i} = \sum_{j} W_{j}^\alpha B_{j} D_{j} f(d_{ij}) \quad & (5a) \\
&B_{j} = \sum_{i} V_{i}^\mu A_{i} O_{i} f(d_{ij}) \quad & (5b)
\end{align}
\]

where

\begin{itemize}
\item
  \(O_{i}\) = an \(n \times 1\) vector of the total number of flows
  emanating from origin \(i\)
\item
  \(D_{j}\) = an \(m \times 1\) vector of the total number of flows
  terminating at destination \(j\)
\item
  \(A_{i}\) = an \(n \times 1\) vector of thevorigin balancing factors
  that ensures the total out-flows are preserved in the predicted flows
\item
  \(B_{j}\) = an \(m \times 1\) vector of the destination balancing
  factors that ensures the total in-flows are preserved in the predicted
  flows
\item
  \(f(d_{ij})\) = a function of cost or distance, referred to as the
  distance-decay function. Most commonly this an exponential or power
  function given by,
\end{itemize}

\[
\begin{align}
&Power\\
&f(d_{ij}) = d_{ij}^\beta \quad & (6) \\
\\
&Exponential \\
&f(d_{ij}) = exp(\beta*d_{ij}) \quad & (7) \\
\end{align}
\]

where \(\beta\) is expected to take a negative value. Different
distance-decay functions assume different responses to higher costs
associated with moving to more distant locations. Of note is that the
unconstrained model with of power function distance-decay is equivalent
to the basic gravity model in equation (2), except that the scaling
factor, \(k\), is not included. In fact, there is no scaling factor in
any of the members of the family of maximum entropy models because there
is a total trip constrained implied in their derivation and subsequently
incorporated into their calibration (Fotheringham \& O'Kelly , 1989).
Another aside is that in the doubly-constrained maximum entoropy model
the values for \(A_{i}\) and \(B_{j}\) are dependent upon each other and
may need to be computed iteratively depending on calibration technqiue.
It is also usually assumed that \(n=m\) for doubly-constrained models.

The family provides different model strucutre depedning on the available
data or the research question at hand. The so-called unconstrained model
does not conserve the total inflows or outflows during parameter
estimation. The production-constrained and attraction-constrained models
conserve either the number of total inflows or outflows, respectively,
and are therefore useful for building models that allocate flows either
to a set of origins or to a set of destinations. Finally, the
doubly-constrained model conserves both the inflows and the outflows at
each location during model calibration. The quantity of explanatory
information provided by each model is given by the number of parameters
it provides. As such, the unconstrained model provides the most
information, followed by the two singly-constrained models, with the
doubly-constrained model providing the least information. Conversely,
the model's predictive power increases with higher quantities of
built-in information (i.e.~total in or out-flows) so that the
doubly-constrained model usually provides the most accurate predictions,
followed by the two singly-constrained models, and the unconstrained
model supplying the weakest predictions (Fotheringham \& O'Kelly, 1989).

    \subsubsection{Calibration framework}\label{calibration-framework}

    Spatial interaction models are often calibrated via linear programming,
nonlinear optimization, or increasingly more often through linear
regression. Given the flexibility and extendability of a regression
framework it was chosen as the primary model calibration technqiue
within the SpInt module. By taking the natural logarithm of both sides
of a spatial interaction model, say the basic gravity model, is is
possible to obtain the so-called log-linear or log-normal spatial
interaction model:

\[\ln{T_{ij}} = k + \mu \ln{V_{i}} + \alpha \ln{W_{j}} - \beta \ln{d_{ij}} + \epsilon \quad (8)\]

where \(\epsilon\) is a a normally distributed error term with a mean of
0. Constrained spatial interaction models can be achieved by including a
fixed effects for the origins (production-constrained), a fixed effect
for the destinations (attraction-constrained) or both
(doubly-constrained). However, there are several limitations of the
log-normal gravity model, which include:

\begin{enumerate}
\def\labelenumi{\arabic{enumi}.}
\itemsep1pt\parskip0pt\parsep0pt
\item
  flows are often counts of people or objects and should be modeled as
  discrete entities;
\item
  flows are often not normally distributed;
\item
  Bias
\item
  zero flows
\end{enumerate}

Therefore, Flowerdew (1984) proposes the Poisson log-linear regression
specification for the family of spatial interaction models. This
specification assumes that the number of flows between \(i\) and \(j\)
is drawn from a Poisson distribution with mean,
\(\lambda_{ij} = T_{ij}\), where \(\lambda_{ij}\) is assumed to be
logarithmically linked to the linear combination of variables,

\[\ln{\lambda_ij} = k + \mu V_{i} + \alpha W_{j} - \beta d_{ij}) \quad (9)\]

and exponentiating both sides of the equation yields the following
family of Poisson log-linear spatial interaction models:

\[
\begin{align}
&Unconstrained \\
&T_{ij} = \exp(k + \mu V_{i} + \alpha W_{j} - \beta d_{ij}) \quad & (10) \\
\\
&Production-constrained \\
&T_{ij} = \exp(k + \mu_{i} + \alpha W_{j} - \beta d_{ij}) \quad & (11) \\
\\
&Attraction-constrained \\
&T_{ij} = \exp(k + \mu V_{i} + \alpha_{j} - \beta d_{ij}) \quad & (12) \\
\\
&Doubly-constrained \\
&T_{ij} = \exp(k + \mu_{i} + \alpha_{j} - \beta d_{ij}) \quad & (13) \\
\end{align}
\]

where \(\mu_{i}\) are origin fixed effects and \(\alpha_{i}\) are
destination fixed effects that achieve the same constraints as the
balancing factors in equations (2-5). Using Poisson regression
automatically alleviates limiations (1) and (2) and since we no longer
need to take the logarithm of \(T_{ij}\), issue (4) is also absolved.
Using fixed effects within Poisson regression to calibrate the
doubly-constrained model also avoids the need for iterative computation
of the balancing factors that exists in other calibration methods.

Calibration of Poisson regression can be carried out within a
generalized linear modeling framework (GLM) using iteratively weighted
least sqaures (IWSL), which converges to the maximum likelihood
estimates for the parameter estimates (Nelder \& Wedderburn, 1972). To
maintain computational efficiency with increasingly larger spatial
interaction datasets, SpInt is built upon a custom GLM/IWLS routine that
leverages sparse data structures for the production-constrained,
attraction-constrained, and doubly-constrained models. As the number of
locations in these models increases, so the does the number of binary
indicator variables needed to construct the fixed effects that enforce
the constraints. Therefore, larger spatial interaction datasets become
increasingly sparse and the utilization of sparse data structures take
advantage of this feature. As a metric, constrained models with
\(n = 3,000\) locations, which implies \(n^2 = 9,000,000\) observations
can still be calibrated within minutes on a standard macbook pro
notebook.

    \subsection{An Illustrative example: migration in
Austria}\label{an-illustrative-example-migration-in-austria}

    \subsubsection{The data}\label{the-data}

    The following example purposefully utilizes data that has was previously
used to demonstrate spatial interaction modeling the R programming
language for validation and consistency. The data is migration flows
between Austrian NUTS level 2 municipalities in 2006. In order to use a
regression-based calibration, the data has to be transformed from the
matrices and vectors described in equations (1-5) to a table where each
row represents a single origin-destination dyad, \((i,j)\) and any
variables associated with either location. Details on how to do this are
outlined further in (LeSage \& Pace, 2008), though this has already been
done in the example data. Let's have a look!

    \begin{Verbatim}[commandchars=\\\{\}]
{\color{incolor}In [{\color{incolor}74}]:} \PY{k+kn}{import} \PY{n+nn}{pandas} \PY{k+kn}{as} \PY{n+nn}{pd}
         \PY{k+kn}{import} \PY{n+nn}{geopandas} \PY{k+kn}{as} \PY{n+nn}{gp}
         \PY{o}{\PYZpc{}}\PY{k}{pylab} inline
         \PY{n}{austria\PYZus{}shp} \PY{o}{=} \PY{n}{gp}\PY{o}{.}\PY{n}{read\PYZus{}file}\PY{p}{(}\PY{l+s+s1}{\PYZsq{}}\PY{l+s+s1}{austria.shp}\PY{l+s+s1}{\PYZsq{}}\PY{p}{)}
         \PY{n}{austria\PYZus{}shp}\PY{o}{.}\PY{n}{plot}\PY{p}{(}\PY{p}{)}
         \PY{n}{austria} \PY{o}{=} \PY{n}{pd}\PY{o}{.}\PY{n}{read\PYZus{}csv}\PY{p}{(}\PY{l+s+s1}{\PYZsq{}}\PY{l+s+s1}{http://dl.dropbox.com/u/8649795/AT\PYZus{}Austria.csv}\PY{l+s+s1}{\PYZsq{}}\PY{p}{)}
         \PY{n}{austria}\PY{o}{.}\PY{n}{drop}\PY{p}{(}\PY{p}{[}\PY{l+s+s1}{\PYZsq{}}\PY{l+s+s1}{Oi2007}\PY{l+s+s1}{\PYZsq{}}\PY{p}{,} \PY{l+s+s1}{\PYZsq{}}\PY{l+s+s1}{Dj2007}\PY{l+s+s1}{\PYZsq{}}\PY{p}{,} \PY{l+s+s1}{\PYZsq{}}\PY{l+s+s1}{OrigAT11}\PY{l+s+s1}{\PYZsq{}}\PY{p}{,} \PY{l+s+s1}{\PYZsq{}}\PY{l+s+s1}{OrigAT12}\PY{l+s+s1}{\PYZsq{}}\PY{p}{,} \PY{l+s+s1}{\PYZsq{}}\PY{l+s+s1}{OrigAT13}\PY{l+s+s1}{\PYZsq{}}\PY{p}{,} \PY{l+s+s1}{\PYZsq{}}\PY{l+s+s1}{OrigAT21}\PY{l+s+s1}{\PYZsq{}}\PY{p}{,} \PY{l+s+s1}{\PYZsq{}}\PY{l+s+s1}{OrigAT22}\PY{l+s+s1}{\PYZsq{}}\PY{p}{,} \PY{l+s+s1}{\PYZsq{}}\PY{l+s+s1}{OrigAT31}\PY{l+s+s1}{\PYZsq{}}\PY{p}{,} \PY{l+s+s1}{\PYZsq{}}\PY{l+s+s1}{OrigAT32}\PY{l+s+s1}{\PYZsq{}}\PY{p}{,} \PY{l+s+s1}{\PYZsq{}}\PY{l+s+s1}{OrigAT33}\PY{l+s+s1}{\PYZsq{}}\PY{p}{,} \PY{l+s+s1}{\PYZsq{}}\PY{l+s+s1}{OrigAT34}\PY{l+s+s1}{\PYZsq{}}\PY{p}{,} \PY{l+s+s1}{\PYZsq{}}\PY{l+s+s1}{DestAT11}\PY{l+s+s1}{\PYZsq{}}\PY{p}{,} \PY{l+s+s1}{\PYZsq{}}\PY{l+s+s1}{DestAT12}\PY{l+s+s1}{\PYZsq{}}\PY{p}{,} \PY{l+s+s1}{\PYZsq{}}\PY{l+s+s1}{DestAT13}\PY{l+s+s1}{\PYZsq{}}\PY{p}{,} \PY{l+s+s1}{\PYZsq{}}\PY{l+s+s1}{DestAT21}\PY{l+s+s1}{\PYZsq{}}\PY{p}{,} \PY{l+s+s1}{\PYZsq{}}\PY{l+s+s1}{DestAT22}\PY{l+s+s1}{\PYZsq{}}\PY{p}{,} \PY{l+s+s1}{\PYZsq{}}\PY{l+s+s1}{DestAT31}\PY{l+s+s1}{\PYZsq{}}\PY{p}{,} \PY{l+s+s1}{\PYZsq{}}\PY{l+s+s1}{DestAT32}\PY{l+s+s1}{\PYZsq{}}\PY{p}{,} \PY{l+s+s1}{\PYZsq{}}\PY{l+s+s1}{DestAT33}\PY{l+s+s1}{\PYZsq{}}\PY{p}{,} \PY{l+s+s1}{\PYZsq{}}\PY{l+s+s1}{DestAT34}\PY{l+s+s1}{\PYZsq{}}\PY{p}{,} \PY{l+s+s1}{\PYZsq{}}\PY{l+s+s1}{beta}\PY{l+s+s1}{\PYZsq{}}\PY{p}{,} \PY{l+s+s1}{\PYZsq{}}\PY{l+s+s1}{Offset}\PY{l+s+s1}{\PYZsq{}}\PY{p}{]}\PY{p}{,} \PY{n}{axis}\PY{o}{=}\PY{l+m+mi}{1}\PY{p}{,} \PY{n}{inplace}\PY{o}{=}\PY{n+nb+bp}{True}\PY{p}{)}
         \PY{n}{austria}\PY{o}{.}\PY{n}{head}\PY{p}{(}\PY{p}{)}
\end{Verbatim}

    \begin{Verbatim}[commandchars=\\\{\}]
Populating the interactive namespace from numpy and matplotlib

    \end{Verbatim}

            \begin{Verbatim}[commandchars=\\\{\}]
{\color{outcolor}Out[{\color{outcolor}74}]:}   Origin Destination  Data    Oi     Dj            Dij
         0   AT11        AT11     0  4016   5146  1.000000e-300
         1   AT11        AT12  1131  4016  25741   1.030018e+02
         2   AT11        AT13  1887  4016  26980   8.420467e+01
         3   AT11        AT21    69  4016   4117   2.208119e+02
         4   AT11        AT22   738  4016   8634   1.320075e+02
\end{Verbatim}
        
    \begin{center}
    \adjustimage{max size={0.9\linewidth}{0.9\paperheight}}{SpIntPrimer_files/SpIntPrimer_11_2.png}
    \end{center}
    { \hspace*{\fill} \\}
    
    The \textbf{Origin} and \textbf{Destination} columns refer to the
origin, \(i\), and destination, \(j\), location labels, the
\textbf{Data} column is the number of flows, the \textbf{Oi} and
\textbf{Dj} columns are the number of total out-flows and total
in-flows, respectively, and the \textbf{Dij} column is the distance
between \(i\) and \(j\). In this case we use the total out-flow and
total in-flow as variables to describe how emissive an origin is and how
attractive a destination is. If we want a more informative model we can
replace these with application specific variables that pertain to
different hypotheses. Next, lets format the data into arrays.

    \begin{Verbatim}[commandchars=\\\{\}]
{\color{incolor}In [{\color{incolor}42}]:} \PY{n}{austria} \PY{o}{=} \PY{n}{austria}\PY{p}{[}\PY{n}{austria}\PY{p}{[}\PY{l+s+s1}{\PYZsq{}}\PY{l+s+s1}{Origin}\PY{l+s+s1}{\PYZsq{}}\PY{p}{]} \PY{o}{!=} \PY{n}{austria}\PY{p}{[}\PY{l+s+s1}{\PYZsq{}}\PY{l+s+s1}{Destination}\PY{l+s+s1}{\PYZsq{}}\PY{p}{]}\PY{p}{]}
         \PY{n}{flows} \PY{o}{=} \PY{n}{austria}\PY{p}{[}\PY{l+s+s1}{\PYZsq{}}\PY{l+s+s1}{Data}\PY{l+s+s1}{\PYZsq{}}\PY{p}{]}\PY{o}{.}\PY{n}{values}
         \PY{n}{Oi} \PY{o}{=} \PY{n}{austria}\PY{p}{[}\PY{l+s+s1}{\PYZsq{}}\PY{l+s+s1}{Oi}\PY{l+s+s1}{\PYZsq{}}\PY{p}{]}\PY{o}{.}\PY{n}{values}
         \PY{n}{Dj} \PY{o}{=} \PY{n}{austria}\PY{p}{[}\PY{l+s+s1}{\PYZsq{}}\PY{l+s+s1}{Dj}\PY{l+s+s1}{\PYZsq{}}\PY{p}{]}\PY{o}{.}\PY{n}{values}
         \PY{n}{Dij} \PY{o}{=} \PY{n}{austria}\PY{p}{[}\PY{l+s+s1}{\PYZsq{}}\PY{l+s+s1}{Dij}\PY{l+s+s1}{\PYZsq{}}\PY{p}{]}\PY{o}{.}\PY{n}{values}
         \PY{n}{Origin} \PY{o}{=} \PY{n}{austria}\PY{p}{[}\PY{l+s+s1}{\PYZsq{}}\PY{l+s+s1}{Origin}\PY{l+s+s1}{\PYZsq{}}\PY{p}{]}\PY{o}{.}\PY{n}{values}
         \PY{n}{Destination} \PY{o}{=} \PY{n}{austria}\PY{p}{[}\PY{l+s+s1}{\PYZsq{}}\PY{l+s+s1}{Destination}\PY{l+s+s1}{\PYZsq{}}\PY{p}{]}\PY{o}{.}\PY{n}{values}
\end{Verbatim}

    The Oi and Dj vectors need not be \(n^2 \times 1\) vectors. In fact,
they can be \(n^2 \times k\) where \(k\) is the number of variables that
are being used to describe origin or desitnation factors associated with
flows.

    \subsubsection{Calibrating the models}\label{calibrating-the-models}

    Now, lets load the main SpInt functions and calibrate some models. The
main SpInt functions are found within the gravity namespace of the SpInt
module ans the estimated parameters can be accessed via the
\textbf{params} attribute of a successfully instantiated spatial
interaction model.

    \begin{Verbatim}[commandchars=\\\{\}]
{\color{incolor}In [{\color{incolor}64}]:} \PY{k+kn}{from} \PY{n+nn}{pysal.contrib.spint.gravity} \PY{k+kn}{import} \PY{n}{Gravity}\PY{p}{,} \PY{n}{Production}\PY{p}{,} \PY{n}{Attraction}\PY{p}{,} \PY{n}{Doubly}
\end{Verbatim}

    Unconstrained (basic gravity) model

    \begin{Verbatim}[commandchars=\\\{\}]
{\color{incolor}In [{\color{incolor}65}]:} \PY{n}{gravity} \PY{o}{=} \PY{n}{Gravity}\PY{p}{(}\PY{n}{flows}\PY{p}{,} \PY{n}{Oi}\PY{p}{,} \PY{n}{Dj}\PY{p}{,} \PY{n}{Dij}\PY{p}{,} \PY{l+s+s1}{\PYZsq{}}\PY{l+s+s1}{exp}\PY{l+s+s1}{\PYZsq{}}\PY{p}{)}
         \PY{k}{print} \PY{n}{gravity}\PY{o}{.}\PY{n}{params}
\end{Verbatim}

    \begin{Verbatim}[commandchars=\\\{\}]
[ 0.44314667  0.51739961 -0.00979932]

    \end{Verbatim}

    Production-constrained model

    \begin{Verbatim}[commandchars=\\\{\}]
{\color{incolor}In [{\color{incolor}66}]:} \PY{n}{production} \PY{o}{=} \PY{n}{Production}\PY{p}{(}\PY{n}{flows}\PY{p}{,} \PY{n}{Origin}\PY{p}{,} \PY{n}{Dj}\PY{p}{,} \PY{n}{Dij}\PY{p}{,} \PY{l+s+s1}{\PYZsq{}}\PY{l+s+s1}{exp}\PY{l+s+s1}{\PYZsq{}}\PY{p}{)}
         \PY{k}{print} \PY{n}{production}\PY{o}{.}\PY{n}{params}\PY{p}{[}\PY{o}{\PYZhy{}}\PY{l+m+mi}{2}\PY{p}{:}\PY{p}{]}
\end{Verbatim}

    \begin{Verbatim}[commandchars=\\\{\}]
[ 0.90285448 -0.0072617 ]

    \end{Verbatim}

    Attraction-constrained model

    \begin{Verbatim}[commandchars=\\\{\}]
{\color{incolor}In [{\color{incolor}54}]:} \PY{n}{attraction} \PY{o}{=} \PY{n}{Attraction}\PY{p}{(}\PY{n}{flows}\PY{p}{,} \PY{n}{Destination}\PY{p}{,} \PY{n}{Oi}\PY{p}{,} \PY{n}{Dij}\PY{p}{,} \PY{l+s+s1}{\PYZsq{}}\PY{l+s+s1}{exp}\PY{l+s+s1}{\PYZsq{}}\PY{p}{)}
         \PY{k}{print} \PY{n}{attraction}\PY{o}{.}\PY{n}{params}\PY{p}{[}\PY{o}{\PYZhy{}}\PY{l+m+mi}{2}\PY{p}{:}\PY{p}{]}
\end{Verbatim}

    \begin{Verbatim}[commandchars=\\\{\}]
[ 0.90037216 -0.00695034]

    \end{Verbatim}

    Doubly-constrained model

    \begin{Verbatim}[commandchars=\\\{\}]
{\color{incolor}In [{\color{incolor}55}]:} \PY{n}{doubly} \PY{o}{=} \PY{n}{Doubly}\PY{p}{(}\PY{n}{flows}\PY{p}{,} \PY{n}{Origin}\PY{p}{,} \PY{n}{Destination}\PY{p}{,} \PY{n}{Dij}\PY{p}{,} \PY{l+s+s1}{\PYZsq{}}\PY{l+s+s1}{exp}\PY{l+s+s1}{\PYZsq{}}\PY{p}{)}
         \PY{k}{print} \PY{n}{doubly}\PY{o}{.}\PY{n}{params}\PY{p}{[}\PY{o}{\PYZhy{}}\PY{l+m+mi}{1}\PY{p}{:}\PY{p}{]}
\end{Verbatim}

    \begin{Verbatim}[commandchars=\\\{\}]
[-0.00791533]

    \end{Verbatim}

    Note that for the constrained models we have limited the params
attribute to print only those associated variables (i.e., not fixed
effects), though it is still possible to access the fixed effect
parameters too.

    \begin{Verbatim}[commandchars=\\\{\}]
{\color{incolor}In [{\color{incolor}67}]:} \PY{k}{print} \PY{n}{production}\PY{o}{.}\PY{n}{params}
\end{Verbatim}

    \begin{Verbatim}[commandchars=\\\{\}]
[-1.16851884  0.52128801  0.98284063 -0.56934181 -0.28515686  0.0381801
 -0.47906115 -0.0141766  -0.1583821   0.90285448 -0.0072617 ]

    \end{Verbatim}

    The first parameter is always the overall intercept with the subsequent
8 parameters representing the fixed effects in this case. You might ask,
``why not 9 fixed effects for the 9 different municipalities?''. Due to
the coding scheme used in SpInt, and many popular statistical
programming languages, you would use \(n - 1\) binary indicator
variables in the design matrix to include the fixed effects for all 9
municipalities in the model. While the non-zero entries in these columns
of the design matrix indicate which rows are associated with which
miunicipality, where a row has all zero entries then refers to the
\(nth\) municipality that has been left out. In Spint this is always the
first origin or destination for the production-constrained and
attraction-constrained models. For the doubly-cosntrained model, both
the first origin and the first destination are left out. In terms of
interpetting the parameters, these dropped locations are assumed to be
0. Since the fixed effects parameters are interpretted as deviations
from the overall intercept, this essentially means the intercept acts as
the fixed effect for the first location. Therefore, we could also say
first 9 parameters are the origin fixed effect parameters and that the
last two parameters are for destination attractiveness and distance,
respectively.

    \subsubsection{Interpretting the
parameters}\label{interpretting-the-parameters}

    \subsubsection{Assessing model fit}\label{assessing-model-fit}

    \begin{Verbatim}[commandchars=\\\{\}]
{\color{incolor}In [{\color{incolor}63}]:} \PY{k}{print} \PY{n}{gravity}\PY{o}{.}\PY{n}{pseudoR2}
         \PY{k}{print} \PY{n}{production}\PY{o}{.}\PY{n}{pseudoR2}
         \PY{k}{print} \PY{n}{attraction}\PY{o}{.}\PY{n}{pseudoR2}
         \PY{k}{print} \PY{n}{doubly}\PY{o}{.}\PY{n}{pseudoR2}
         \PY{k}{print} \PY{l+s+s1}{\PYZsq{}}\PY{l+s+s1}{\PYZsq{}}
         \PY{k}{print} \PY{n}{gravity}\PY{o}{.}\PY{n}{D2}
         \PY{k}{print} \PY{n}{production}\PY{o}{.}\PY{n}{D2}
         \PY{k}{print} \PY{n}{attraction}\PY{o}{.}\PY{n}{D2}
         \PY{k}{print} \PY{n}{doubly}\PY{o}{.}\PY{n}{D2}
         \PY{k}{print} \PY{l+s+s2}{\PYZdq{}}\PY{l+s+s2}{\PYZdq{}}
         \PY{k}{print} \PY{n}{gravity}\PY{o}{.}\PY{n}{SSI}
         \PY{k}{print} \PY{n}{production}\PY{o}{.}\PY{n}{SSI}
         \PY{k}{print} \PY{n}{attraction}\PY{o}{.}\PY{n}{SSI}
         \PY{k}{print} \PY{n}{doubly}\PY{o}{.}\PY{n}{SSI}
         \PY{k}{print} \PY{l+s+s2}{\PYZdq{}}\PY{l+s+s2}{\PYZdq{}}
         \PY{k}{print} \PY{n}{gravity}\PY{o}{.}\PY{n}{SRMSE}
         \PY{k}{print} \PY{n}{production}\PY{o}{.}\PY{n}{SRMSE}
         \PY{k}{print} \PY{n}{attraction}\PY{o}{.}\PY{n}{SRMSE}
         \PY{k}{print} \PY{n}{doubly}\PY{o}{.}\PY{n}{SRMSE}
\end{Verbatim}

    \begin{Verbatim}[commandchars=\\\{\}]
0.812858632041
0.910155702595
0.909354566583
0.943539912198

0.834893355219
0.913167247331
0.912363460094
0.946661920896

0.68757357636
0.740913740348
0.752155260541
0.811852110904

1.01757072219
0.464519773826
0.58404798182
0.37928618533

    \end{Verbatim}

    \subsubsection{Local models}\label{local-models}

    \subsubsection{Testing for
overdispersion}\label{testing-for-overdispersion}

    \subsection{Additional functionality}\label{additional-functionality}

    \subsubsection{Existing features}\label{existing-features}

    \subsubsection{Future additions}\label{future-additions}

    Finally, there are several paradigms for incorporating spatial effects
into spatial interaction models (competing destinations, spatial
autoregressive, eigenvector spatial filter)

    \begin{Verbatim}[commandchars=\\\{\}]
{\color{incolor}In [{\color{incolor} }]:} 
\end{Verbatim}


    % Add a bibliography block to the postdoc
    
    
    
    \end{document}
